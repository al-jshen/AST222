\documentclass[11pt,letterpaper]{article}
\usepackage{fullpage}
\usepackage[top=0.5in, bottom=1.5in, left=1in, right=1in]{geometry}
\usepackage{amsmath,amsthm,amsfonts,amssymb,amscd}
\usepackage{lastpage}
\usepackage{enumerate}
\usepackage{enumitem}
\usepackage{fancyhdr}
\usepackage{graphicx}
\usepackage{listings}
\usepackage{hyperref}
\usepackage{booktabs}
\usepackage{cancel}
\usepackage{physics}
\usepackage{caption,cleveref,colortbl,csquotes,datatool,helvet,mathpazo,multirow,listings,pgfplots,xcolor}

\hypersetup{%
  colorlinks=true,
  linkcolor=blue,
  linkbordercolor={0 0 1}
}

\setlength{\parindent}{0.0in}
\setlength{\parskip}{0.05in}
\setlength{\footnotesep}{1.2\baselineskip}

% edit these
\newcommand\course{AST222H}
\newcommand\Title{Problem Set 4}
\newcommand\Name{Jeff Shen} 
\newcommand\Id{1004911526} 
\newcommand\Date{20 Mar 2020}

\pagestyle{fancyplain}
\headheight 35pt
\lhead{\Name}
\lhead{\Name\\\Id}
\chead{\LARGE \Title}
\rhead{\course \\ \Date}
\lfoot{}
\cfoot{}
\rfoot{\small\thepage}
\pgfplotsset{compat=1.16}
\headsep 1.2em

\begin{document}

\section*{Problem 1}

The spectrum would depend on the angle between the observer and the rotational axis of the AGN. Narrow emission lines come from lower-velocity material that is more distant from the center of the AGN, and are visible regardless of whether the angle is large or small. Broad emission lines are caused by Doppler broadening due to the higher velocities of material closer to the center. These are only visible if the angle is small, because at larger angles, the torus of dust/gas surrounding the center blocks the view of the center. 

\section*{Problem 2}

Radio-loud AGN have jets/lobes, while radio-quiet AGN do not. These jets are caused by relativistic particles accelerated out of the center of the AGN. The particles emit synchrotron radiation because of strong magnetic fields, and synchrotron emissions peak in the radio regime. 

\section*{Problem 3}

In free-fall, kinetic energy is given by 
\begin{align*}
    E_k = \frac{1}{2}mv^2,
\end{align*}
where $m$ is the mass of the parcel and $v$ is the velocity of the parcel. Gravitational potential energy is given by 
\begin{align*}
    E_g = -\frac{GMm}{r},
\end{align*}
where $M$ and $r$ are the mass and radius of the neutron star, respectively. In free-fall, energy is conserved, and 
\begin{align*}
    \frac{1}{2}mv^2 = \frac{GMm}{r}.
\end{align*}
We also know that luminosity is energy emitted per unit time, which is the time derivative of the above. $M$ and $r$ are constant, so on the right, we take the time derivative of $m$:
\begin{align*}
    L = \frac{1}{2}\dot{m}v^2 = \frac{GM\dot{m}}{r}.
\end{align*}
We want an $mc^2$ term, so we can multiply both the numerator and denominator by $c^2$:
\begin{align*}
    L = \frac{GM\dot{m}}{r}\times \frac{c^2}{c^2} = \frac{GM}{rc^2}\times \dot{m}c^2.
\end{align*}
We can call the first term (the coefficient on the $\dot{m}c^2$) the efficiency term $\eta$. For the parameters given in the question, we find that 
\begin{align*}
    \eta = \frac{GM}{rc^2} = \frac{6.67\times 10^{-11}~{\rm m^3\,kg^{-1}\times 1.4\times 1.989\times 10^{30}~{\rm kg}}}{10^4~{\rm m}\times (2.99\times 10^{8}~{\rm m\,s^{-1}})^2} = 0.207,
\end{align*}
which is the fraction of the rest-mass energy energy released. 

\section*{Problem 4}

\begin{enumerate}[label=(\roman*)]
    \item The rest wavelength of the $H\alpha$ emission line in a vacuum (?) is $6564.614~{\rm \r{A}} = 0.656~{\mu m}$\footnote{https://classic.sdss.org/dr3/products/spectra/vacwavelength.html}. The redshift of the source is 
        \begin{align*}
            z = \frac{4~{\mu m} - 0.656~{\mu m}}{0.656~{\mu m}} = 5.01.
        \end{align*}

    \item We can relate the velocity dispersion to the width of the emission line using the following equation:
        \begin{align*}
            \frac{\Delta \lambda}{\lambda} = \frac{\Delta v}{c}.
        \end{align*}
        Rearranging for $\Delta v$ and plugging in the appropriate values, we find that $\Delta v$ is {\huge is $\Delta \lambda$ supposed to be the width (0.025)? or the difference (4-0.656) from last q?}:
        \begin{align*}
            \Delta v = \frac{\Delta \lambda}{\lambda}\times c = \frac{0.025~{\mu m}}{0.656~{\mu m}}\times 2.99\times 10^{8}~{\rm m\,s^{-1}} = ~{\rm m\,s^{-1}}.
        \end{align*}
        Yes, this is fairly consistent with a $2000~{\rm km\,s^{-1}}$ velocity dispersion.
\end{enumerate}

\section*{Problem 5}

The mass of Sgr A* is approximately $4\times 10^{6}~{\rm M_\odot}$\footnote{https://arxiv.org/pdf/1607.05726.pdf}. The Schwarzchild radius is
\begin{align*}
    R_S = \frac{2GM}{c^2} = \frac{2\times (6.67\times 10^{-11}~{\rm m^3\,kg^{-1}\,s^{-2}})\times (4\times 10^{6}~{\rm M_\odot})\times (1.989\times 10^{30}~{\rm kg\,M_\odot^{-1}})}{(2.99\times 10^{8}~{\rm m\,s^{-1}})^2} = 1.18\times 10^{10}~{\rm m}.
\end{align*}
The distance between the Sun and the center of the Milky Way is $8~{\rm kpc} = 2.47\times 10^{20}~{\rm m}$. We can use this, along with the small angle approximation, to convert the Schwarzchild radius of Sgr A* to an angular size (which is the resolution required):
\begin{align*}
    \theta \simeq \tan{\theta} = \frac{R_s}{d} = \frac{1.18\times 10^{10}~{\rm m}}{2.47\times 10^{20}~{\rm m}} = 4.78\times 10^{-11}~{\rm rad} \simeq 0.01~{\rm mas}. 
\end{align*}
If VLBI is used and the effective diameter of the "telescope" is the diameter of the Earth, then the wavelength that would have to be used is 
\begin{alignat*}{2}
    &&\theta &= \frac{1.22\lambda}{2R_E} \\
    \implies&&\lambda &= \frac{\theta \times 2R_E}{1.22} \\
    &&&= \frac{4.78\times 10^{-11}~{\rm rad}\times 2\times 6.378\times 10^{6}~{\rm m}}{1.22} \\
    &&&= 0.0005~{\rm m}.
\end{alignat*}


\section*{Problem 6}

C\&O Equation 26.2:
\begin{align*}
    t_c = \frac{2\pi v_M r_i^2}{CGM}
\end{align*}
where $t_c$ is the time required for the cluster to spiral in to the center of the Milky Way, $v_M$ is the orbital speed ($\sim 220~{\rm km\,s^{-1}}$), $r_i$ is the initial distance of the cluster from the center, and $M$ is the mass of the cluster. Plugging in the appropriate values and performing the necessary unit conversions, we get 
\begin{align*}
    t_c = \frac{2\pi \times 220~{\rm km\,s^{-1}}\times 5~{\rm kpc}}{75\times 6.67\times 10^{-11}~{\rm m^3\,kg^{-1}}\times 5\times 10^{6}~{\rm M_\odot}} = 20.95~{\rm Gyr}.
\end{align*}
This means that dynamical friction is not relevant for Milky Way globular clusters, as the time required to sink to the center is greater than the age of the Universe. 

\section*{Problem 7}

\begin{enumerate}[label=(\roman*)]
    \item We can do some algebra and find $g(t)$:
        \begin{alignat}{2}
            &&Z(t) &= -y\ln{\left(\frac{g(t)}{M_b}\right)} = Z_x\\
            \implies&&-\frac{Z_x}{y} &= \ln{\left(\frac{g(t)}{M_b}\right)} \\
            \implies&&e^{-Z_x/y} &= \frac{g(t)}{M_b} \\
            \implies&&g(t) &= e^{-Z_x/y}M_b.
        \end{alignat}
    Using this, we can write $s(t)$ in terms of $M_b$ and $Z_x$:
    \begin{align}
        s(t) &= M_b - g(t) \\
        &= M_b - e^{-Z_x/y}M_b \\
        &= M_b(1-e^{-Z_x/y}).
    \end{align}
    
    \item some description here
    \begin{align}
        \frac{s_{Z<Z_\odot/3}}{s_{Z<Z_\odot}} &= \frac{M(1-e^{-Z_\odot/3y})}{M(1-e^{-Z_\odot/y})}
    \end{align}
    From Eq. 3 above, we know that $e^{-Z_\odot/y} = g/M_b$. Similarly, we can rewrite the $e^{-Z_\odot/3y}$ term by starting from Eq. 2:
    \begin{alignat}{2}
        &&-\frac{Z_x}{y} &= \ln{\left(\frac{g(t)}{M_b}\right)} \\
        \implies&&-\frac{Z_x}{3y} &= \frac{1}{3}\ln{\left(\frac{g(t)}{M_b}\right)} = \ln{\left(\left(\frac{g(t)}{M_b}\right)^{1/3}\right)} \\
        \implies&&e^{-Z_\odot/y} &= \left(\frac{g(t)}{M_b}\right)^{1/3}.
    \end{alignat}
    Then we can rewrite Eq. 8 as 
    \begin{align}
        \frac{1-(g/M_b)^{1/3}}{1-(g/M_b)}.
    \end{align}
    For $g/M_b = 0.1$, we have 
    \begin{align}
        \frac{1-0.1^{1/3}}{1-0.1} \simeq 0.5954.
    \end{align}
\end{enumerate}


\end{document}
