\documentclass[11pt,letterpaper]{article}
\usepackage{fullpage}
\usepackage[top=2cm, bottom=4.5cm, left=2.5cm, right=2.5cm]{geometry}
\usepackage{amsmath,amsthm,amsfonts,amssymb,amscd}
\usepackage{lastpage}
\usepackage{enumerate}
\usepackage{enumitem}
\usepackage{fancyhdr}
\usepackage{graphicx}
\usepackage{listings}
\usepackage{hyperref}
\usepackage{booktabs}
\usepackage{cancel}
\usepackage{caption,cleveref,colortbl,csquotes,datatool,helvet,mathpazo,multirow,listings,pgfplots,xcolor}

\hypersetup{%
  colorlinks=true,
  linkcolor=blue,
  linkbordercolor={0 0 1}
}

\setlength{\parindent}{0.0in}
\setlength{\parskip}{0.05in}
\setlength{\footnotesep}{1.2\baselineskip}


% edit these
\newcommand\course{AST222H}
\newcommand\Title{Problem Set 1}
\newcommand\Name{Jeff Shen} 
\newcommand\Id{1004911526} 
\newcommand\Date{22 Jan 2020}

\pagestyle{fancyplain}
\headheight 35pt
\lhead{\Name}
\lhead{\Name\\\Id}
\chead{\LARGE \Title}
\rhead{\course \\ \Date}
\lfoot{}
\cfoot{}
\rfoot{\small\thepage}
\pgfplotsset{compat=1.16}
\headsep 1.2em

\begin{document}

% problem 1
\section*{Problem 1}
\begin{enumerate}[label=(\alph*)]

\item We can use the formula for angular resolution (with $d=R_E=6.378\times 10^{6}~{\rm m}$) to calculate this:
    \begin{align*}
        \theta = \frac{1.22\lambda}{d} = \frac{1.22\times 0.21~{\rm m}}{6.378\times 10^{6}~{\rm m}} = 4.02\times 10^{-8}~{\rm rad}
    \end{align*} 

\item The effective diameter of the telescope would be increased to the distance between the Earth and the Moon. {\huge which side of Earth/Moon? does telescope work on far side of the moon?}Using the same equation as above, the angular resolution would be 
    \begin{align*}
        \theta = \frac{1.22\lambda}{d} = \frac{1.22\times 0.21~{\rm m}}{3.844\times 10^{8}~{\rm m}} = 6.66\times 10^{-10}~{\rm rad}.
    \end{align*}
    Comparing this to the previous result, the angular resolution is increased by a factor of 
    \begin{align*}
        \frac{\theta_1}{\theta_2} = \frac{4.02\times 10^{-8}~{\rm rad}}{6.66\times 10^{-10}~{\rm rad}} = 60.3~{\rm x}.
    \end{align*}

\end{enumerate}

% problem 2
\section*{Problem 2}
\begin{enumerate}[label=(\alph*)]
        
    \item The absolute magnitude of the star can be found using the distance modulus formula 
        \begin{align*}
            m - M = 5\log(d) - 5,
        \end{align*}
        where $m$ is the apparent magnitude, $M$ is the absolute magnitude, and $d$ is the distance to the star in parsecs. Then we find that the absolute magnitude is
        \begin{align*}
            M = m - 5\log(d) + 5 = 21 - 5\log(3000) + 5 = 21 - 17.4 + 5 = 8.6.
        \end{align*}
        The stellar type of Delorean would probably be M \footnote{https://sites.uni.edu/morgans/astro/course/Notes/section2/spectralmasses.html}.

    \item We can rearrange the distance modulus equation, accounting for reddening, to isolate absolute magnitude:
        \begin{alignat*}{2}
            &&d &= 10^{(m - M + 5 - A) / 5} \\
            \implies&&\log(d) &= (m-M+5-A)/5 \\
            \implies&&M &= m+5-A - 5\log(d)
        \end{align*}
        The difference is just the $-A$ term, so we can avoid any calculations by simply observing that we can subtract {\huge units? is mag/pc the right one? how to properly estimate spectral type?} the reddening value from our previous result.
        \begin{itemize}
            \item For a reddening value of 1 mag/pc, $M=7.6$. This would probably be type K.
            \item For a reddening value of 2 mag/pc, $M=6.6$. This would probably be type K.
            \item For a reddening value of 3 mag/pc, $M=5.6$. This would probably be type G. 
        \end{itemize}
        Reddening does make a difference in the estimation of Delorean's stellar type. 

    \item 
\end{enumerate}


% problem 3
\section*{Problem 3}
% problem 4
\section*{Problem 4} 

\begin{enumerate}[label=(\alph*)]
\end{enumerate}


\end{document}
