\documentclass[11pt,letterpaper]{article}
\usepackage{fullpage}
\usepackage[top=2cm, bottom=4.5cm, left=2.5cm, right=2.5cm]{geometry}
\usepackage{amsmath,amsthm,amsfonts,amssymb,amscd}
\usepackage{lastpage}
\usepackage{enumerate}
\usepackage{enumitem}
\usepackage{fancyhdr}
\usepackage{graphicx}
\usepackage{listings}
\usepackage{hyperref}
\usepackage{booktabs}
\usepackage{cancel}
\usepackage{caption,cleveref,colortbl,csquotes,datatool,helvet,mathpazo,multirow,listings,pgfplots,xcolor}

\hypersetup{%
  colorlinks=true,
  linkcolor=blue,
  linkbordercolor={0 0 1}
}

\setlength{\parindent}{0.0in}
\setlength{\parskip}{0.05in}
\setlength{\footnotesep}{1.2\baselineskip}


% edit these
\newcommand\course{AST222H}
\newcommand\Title{Problem Set 1}
\newcommand\Name{Jeff Shen} 
\newcommand\Id{1004911526} 
\newcommand\Date{22 Jan 2020}

\pagestyle{fancyplain}
\headheight 35pt
\lhead{\Name}
\lhead{\Name\\\Id}
\chead{\LARGE \Title}
\rhead{\course \\ \Date}
\lfoot{}
\cfoot{}
\rfoot{\small\thepage}
\pgfplotsset{compat=1.16}
\headsep 1.2em

\begin{document}

% problem 1
\section*{Problem 1}
\begin{enumerate}[label=(\alph*)]

\item We can use the formula for angular resolution (with $d=R_E=6.378\times 10^{6}~{\rm m}$) to calculate this:
    \begin{align*}
        \theta = \frac{1.22\lambda}{d} = \frac{1.22\times 0.21~{\rm m}}{6.378\times 10^{6}~{\rm m}} = 4.02\times 10^{-8}~{\rm rad}
    \end{align*} 

\item The effective diameter of the telescope would be increased to the distance between the Earth and the Moon. Taking that distance to be the mean distance (ignoring whether we use the close or the far side of the Moon, since that is negligible) and the same equation as above, the angular resolution would be 
    \begin{align*}
        \theta = \frac{1.22\lambda}{d} = \frac{1.22\times 0.21~{\rm m}}{3.844\times 10^{8}~{\rm m}} = 6.66\times 10^{-10}~{\rm rad}.
    \end{align*}
    Comparing this to the previous result, the angular resolution is increased by a factor of 
    \begin{align*}
        \frac{\theta_1}{\theta_2} = \frac{4.02\times 10^{-8}~{\rm rad}}{6.66\times 10^{-10}~{\rm rad}} = 60.3~{\rm x}.
    \end{align*}

\end{enumerate}

% problem 2
\section*{Problem 2}
\begin{enumerate}[label=(\alph*)]
        
    \item The absolute magnitude of the star can be found using the distance modulus formula 
        \begin{align*}
            m - M = 5\log(d) - 5,
        \end{align*}
        where $m$ is the apparent magnitude, $M$ is the absolute magnitude, and $d$ is the distance to the star in parsecs. Then we find that the absolute magnitude is
        \begin{align*}
            M = m - 5\log(d) + 5 = 21 - 5\log(3000) + 5 = 21 - 17.4 + 5 = 8.6.
        \end{align*}
        Assuming that Delorean is a MS star, its stellar type would probably be M \footnote{https://sites.uni.edu/morgans/astro/course/Notes/section2/spectralmasses.html}.

    \item We can rearrange the distance modulus equation, accounting for reddening, to isolate absolute magnitude:
        \begin{alignat*}{2}
            &&d &= 10^{(m - M + 5 - A) / 5} \\
            &&&= 10^{(m-M+5-kd)/5} \\
            \implies&&\log(d) &= (m-M+5-kd)/5 \\
            \implies&&M &= m+5-kd - 5\log(d)
        \end{align*}
        \begin{itemize}
            \item For a reddening value of $k=10^{-3}~{\rm mag/pc}$, the absolute magnitude is 
                \begin{align*}
                    M = 21 + 5 - 10^{-3}~{\rm mag/pc}\times 3000~{\rm pc}- 5\log(3000) = 21 + 5 - 3 - 17.4 = 5.6.
                \end{align*}
                This would probably be a type G star. 
            \item For a reddening value of $k=2\times 10^{-3}~{\rm mag/pc}$, the absolute magnitude is 
                \begin{align*}
                    M = 21 + 5 - 2\times 10^{-3}~{\rm mag/pc}\times 3000~{\rm pc}- 5\log(3000) = 21 + 5 - 6 - 17.4 = 2.6.
                \end{align*}
                This would probably be a type A star. 
            \item For a reddening value of $k=3\times 10^{-3}~{\rm mag/pc}$, the absolute magnitude is 
                    \begin{align*}
                        M = 21 + 5 - 3\times 10^{-3}~{\rm mag/pc}\times 3000~{\rm pc}- 5\log(3000) = 21 + 5 - 9 - 17.4 = -0.4.
                \end{align*}
                This would probably be a type B star. 
        \end{itemize}
        Reddening does make a difference in the estimation of Delorean's stellar type. 

    \item Since Delorean is in the galactic plane, we can ignore $z$, and model the density of the Milky Way as follows \footnote{taken from page 24 of week 2 lecture slides}:
        \begin{align*}
            \rho(R) = \rho_0\,e^{-(R-R_0)/h_R},
        \end{align*}
        where $\rho_0 = 0.04~{\rm M_\odot\,pc^{-3}}$, $R_0 = 8}~{\rm kpc}$, $h_R = 2.5~{\rm kpc}$, and $R$ is the distance from the center of the galaxy in kpc. Since Delorean is 3 kpc from Earth, Earth is 3 kpc from the center of the galaxy, and Delorean is located between Earth and the center of the galaxy, Delorean is 5 kpc from the center of the galaxy. In that region, the density of the Milky Way is 
        \begin{align*}
            \rho(5) = 0.04~{\rm M_\odot\,pc^{-3}}\,e^{-(5~{\rm kpc}-8~{\rm kpc})/2.5~{\rm kpc}} = 0.133~{\rm M_\odot\,pc^{-3}}.
        \end{align*}
        We can use this to compare the dynamical times of the Sun and Delorean:
        \begin{alignat*}{2}
            &&\frac{t_S}{t_D} &= \frac{\frac{3}{\sqrt{G\overline{\rho_S}}}}{\frac{3}{\sqrt{G\overline{\rho_D}}}} = \frac{\sqrt{\overline{\rho_D}}}{\sqrt{\overline{\rho_S}}} \\
            \implies&&t_D &= t_S\,\frac{\sqrt{\overline{\rho_S}}}{\sqrt{\overline{\rho_D}}} \\
        \end{alignat*}
            Knowing that the dynamical time of the Sun is about 230 Myr, we can find the dynamical time of Delorean:
        \begin{align*}
            t_D = 230~{\rm Myr}\,\frac{\sqrt{0.04~{\rm M_\odot\,pc^{-3}}}}{\sqrt{0.133~{\rm M_\odot\,pc^{-3}}}} \simeq 126~{\rm Myr}.
        \end{align*}
    \item The Sun is 8 kpc away from the center of the galaxy, Delorean is 3 kpc away from the Sun, and the Sun is directly in between Delorean and the center of the galaxy. So Delorean is 11 kpc away from the center of the galaxy. Performing the same calculations as above, the density of Delorean is 
        \begin{align*}
            \rho(11) = 0.04~{\rm M_\odot\,pc^{-3}}\,e^{-(11~{\rm kpc}-8~{\rm kpc})/2.5~{\rm kpc}} = 0.012~{\rm M_\odot\,pc^{-3}},
        \end{align*}
        and its dynamical time is 
        \begin{align*}
            t_D = 230~{\rm Myr}\,\frac{\sqrt{0.04~{\rm M_\odot\,pc^{-3}}}}{\sqrt{0.012~{\rm M_\odot\,pc^{-3}}}} \simeq 420~{\rm Myr}.
        \end{align*}

\end{enumerate}


% problem 3
\section*{Problem 3}
{\huge you want to point the telescope away from the galactic center since you cant see through it, and the brightness will interfere with pics}
We can convert the coordinates of the Hubble Deep Field to galactic coordinates:  
\begin{align*}
    l = (3*3600 + 32*60 + 39) / (24*3600) * 360 = 53.16^\circ \\
    b = 
\end{align*}
% problem 4
\section*{Problem 4} 

\begin{enumerate}[label=(\alph*)]
        The acceleration of the star is given by the equation 
        \begin{align*}
            a = \frac{v^2}{r},
        \end{align*}
        and we know that this must be equivalent to the acceleration due to the gravity of the mass enclosed within its orbital radius, which is 
        \begin{align*}
            a = \frac{GM(r)}{r^2},
        \end{align*}
        where $M(r)$ is the mass enclosed as a function of radius.
        Equating these two, we find that we can express the mass enclosed as 
        \begin{alignat*}{2}
            &&\frac{v^2}{r} &= \frac{GM(r)}{r^2} \\
            \implies&&M(r) &= \frac{v^2r}{G}.
        \end{alignat*}
        If we want a flat rotation curve, we must show that $v$ is constant (not proportional to radius). This is equivalent to showing that the mass is directly proportional to the radius: $M \propto r$. 

        Note also that the mass (assuming spherical symmetry) is can be expressed in terms of the density as 
        \begin{align*}
            M = \rho V,
        \end{align*}
        where the volume V is given by 
        \begin{align*}
            V = \frac{4}{3}\pi r^3 \propto r^3.
        \end{align*}
    \item If we start with $\rho \propto 1/r^{3.5}$ and try to find the how the mass enclosed and the radius are related, we find that 
        \begin{align*}{2}
            M = \rho V \propto (\frac{1}{r^{3.5}}) (r^{3}) \propto \frac{1}{r^{0.5}},
        \end{align*}
        which is not what we wanted. This means that this density profile does not produce a flat rotation curve. 
    \item We repeat the same process above, but with $\rho \propto 1/r^{2}$:
        \begin{align*}{2}
            M = \rho V \propto (\frac{1}{r^{2}}) (r^{3}) \propto \frac{1}{r},
        \end{align*}
        which shows that this density profile produces a flat rotation.
\end{enumerate}

% problem 5
\section*{Problem 5} 

\begin{enumerate}[label=(\alph*)]
    \item In this problem, $r_0$ will be used as the initial distance instead of $r$. The work required to move a body over a distance $dr$ against the force of gravity is given by 
        \begin{align*}
            W = Fdr = \frac{GMm}{r^2}dr,
        \end{align*}
        where $M$ is the mass of the point mass galaxy, $m$ is the mass of the body being moved, and $r$ is the distance between the two. To escape the gravitational force of the galaxy, the object must be moved to an infinite distance. The energy required to do this can be calculated as follows:
        \begin{align*}
            \int_{r_0}^{\infty} \frac{GMm}{r^2}dr &= GMm \int_{r_0}^{\infty} r^{-2}dr \\
            &= GMm\left(-\frac{1}{r}\right)\Big|_{r_0}^{\infty} \\
            &= -GMm\left(\frac{1}{\infty} - \frac{1}{r_0}\right) \\
            &= -GMm\left(-\frac{1}{r_0}\right) \\
            &= \frac{GMm}{r_0}.
        \end{align*}
        If we consider putting this energy into the body as kinetic energy, it will have a (squared) velocity of 
        \begin{alignat*}{2}
            &&\frac{1}{2}mv^2 &= \frac{GMm}{r_0} \\
            \implies&&v^2 &= \frac{2GM\cancel{m}}{r_0\cancel{m}} \\
            \implies&&v^2 &= \frac{2GM}{r_0}.
        \end{align*}
\end{enumerate}

% problem 6
\section*{Problem 6} 

\begin{enumerate}[label=(\alph*)]
\end{enumerate}

% problem 7
\section*{Problem 7} 

\begin{enumerate}[label=(\alph*)]
\end{enumerate}




\end{document}
